%
% The first command in your LaTeX source must be the \documentclass command.
\documentclass[sigconf]{acmart}
%\documentclass{acmart}

%
% defining the \BibTeX command - from Oren Patashnik's original BibTeX documentation.
\def\BibTeX{{\rm B\kern-.05em{\sc i\kern-.025em b}\kern-.08emT\kern-.1667em\lower.7ex\hbox{E}\kern-.125emX}}
    
% Rights management information. 
% This information is sent to you when you complete the rights form.
% These commands have SAMPLE values in them; it is your responsibility as an author to replace
% the commands and values with those provided to you when you complete the rights form.
%
% These commands are for a PROCEEDINGS abstract or paper.
%\copyrightyear{2018}
%\acmYear{2018}
%\setcopyright{acmlicensed}
%\acmConference[Woodstock '18]{Woodstock '18: ACM Symposium on Neural Gaze Detection}{June 03--05, 2018}{Woodstock, NY}
% \acmBooktitle{Woodstock '18: ACM Symposium on Neural Gaze Detection, June 03--05, 2018, Woodstock, NY}
% \acmPrice{15.00}
% \acmDOI{10.1145/1122445.1122456}
% \acmISBN{978-1-4503-9999-9/18/06}

%
% These commands are for a JOURNAL article.
%\setcopyright{acmcopyright}
%\acmJournal{TOG}
%\acmYear{2018}\acmVolume{37}\acmNumber{4}\acmArticle{111}\acmMonth{8}
%\acmDOI{10.1145/1122445.1122456}

%
% Submission ID. 
% Use this when submitting an article to a sponsored event. You'll receive a unique submission ID from the organizers
% of the event, and this ID should be used as the parameter to this command.
%\acmSubmissionID{123-A56-BU3}

%
% The majority of ACM publications use numbered citations and references. If you are preparing content for an event
% sponsored by ACM SIGGRAPH, you must use the "author year" style of citations and references. Uncommenting
% the next command will enable that style.
%\citestyle{acmauthoryear}

%
% end of the preamble, start of the body of the document source.
\begin{document}

%\bibliography{/Users/ClayElmore/Desktop/mendeley/DMD_Literature.bib}

%
% The "title" command has an optional parameter, allowing the author to define a "short title" to be used in page headers.
\title{Stochastic Energy Market Price Forecasting with Recurrent Neural Networks}

%
% The "author" command and its associated commands are used to define the authors and their affiliations.
% Of note is the shared affiliation of the first two authors, and the "authornote" and "authornotemark" commands
% used to denote shared contribution to the research.

\author{Elmore, Clay}
%\authornote{Both authors contributed equally to this research.}
\email{celmore1@nd.edu}
\affiliation{%
  \institution{University of Notre Dame}
}

\author{Kopp, Grace}
\email{gkopp@nd.edu}
\affiliation{%
  \institution{University of Notre Dame}
}
%
% By default, the full list of authors will be used in the page headers. Often, this list is too long, and will overlap
% other information printed in the page headers. This command allows the author to define a more concise list
% of authors' names for this purpose.
\renewcommand{\shortauthors}{Elmore and Kopp, et al.}

% ============================================================================================================================= %
%														Abstract																       %
% ============================================================================================================================= %
% The abstract is a short summary of the work to be presented in the article.
\begin{abstract}
This project proposes to construct a neural network that will forecast energy prices in the California Independent System Operator (CAISO). Accurate forecasting of energy prices in the CISO provides an economic opportunity for energy providers to incur profit and buyers to purchase at the lowest possible price. Traditional methods for forecasting have been used for many years to forecast these prices, but the recent advances in recurrent neural network time series forecasting presents a possible way to improve upon existing methods. The problem statement of this project is how to design a neural network to accurately predict stochastic energy prices.
\end{abstract}

%
% The code below is generated by the tool at http://dl.acm.org/ccs.cfm.
% Please copy and paste the code instead of the example below.
%
% \begin{CCSXML}
% <ccs2012>
%  <concept>
%   <concept_id>10010520.10010553.10010562</concept_id>
%   <concept_desc>Computer systems organization~Embedded systems</concept_desc>
%   <concept_significance>500</concept_significance>
%  </concept>
%  <concept>
%   <concept_id>10010520.10010575.10010755</concept_id>
%   <concept_desc>Computer systems organization~Redundancy</concept_desc>
%   <concept_significance>300</concept_significance>
%  </concept>
%  <concept>
%   <concept_id>10010520.10010553.10010554</concept_id>
%   <concept_desc>Computer systems organization~Robotics</concept_desc>
%   <concept_significance>100</concept_significance>
%  </concept>
%  <concept>
%   <concept_id>10003033.10003083.10003095</concept_id>
%   <concept_desc>Networks~Network reliability</concept_desc>
%   <concept_significance>100</concept_significance>
%  </concept>
% </ccs2012>
% \end{CCSXML}

% \ccsdesc[500]{Computer systems organization~Embedded systems}
% \ccsdesc[300]{Computer systems organization~Redundancy}
% \ccsdesc{Computer systems organization~Robotics}
% \ccsdesc[100]{Networks~Network reliability}

%
% Keywords. The author(s) should pick words that accurately describe the work being
% presented. Separate the keywords with commas.
\keywords{machine learning, stochastic forecasting, neural networks, energy markets}

%
% A "teaser" image appears between the author and affiliation information and the body 
% of the document, and typically spans the page. 
% \begin{teaserfigure}
%   \includegraphics[width=\textwidth]{sampleteaser}
%   \caption{Seattle Mariners at Spring Training, 2010.}
%   \Description{Enjoying the baseball game from the third-base seats. Ichiro Suzuki preparing to bat.}
%   \label{fig:teaser}
% \end{teaserfigure}

%
% This command processes the author and affiliation and title information and builds
% the first part of the formatted document.
\maketitle

% ============================================================================================================================= %
%														Intro  																       %
% ============================================================================================================================= %

\section{Introduction}

%Outline:
%\begin{itemize}
%	\item Large population of California depends on energy market every day\cite{Dowling2017}.
%	\item Problem is interesting because it is data rich, and quite complex.
%	\item The basic economic principle of buy-low sell-high allows a natural economic opportunity that arises to participate in the energy market with accurate forecasts.
%	\item Modern advances in neural networks have shown good results in time-series forecasting (references).
%	\item For the previous reason, we believe we can use neural networks to predict energy prices.
%	\item Problem statement: how to construct a neural network that accurately predicts stochastic, periodic prices in the CISO.
%	\item To answer this problem statement we will:
%	\begin{itemize}
%		\item Build a time-series forecasting neural network.
%		\item Benchmark the NN against existing research.
%		\item We expect that a NN will perform at a higher level than existing traditional methods. 
%	\end {itemize}
%\end{itemize}

The California Independent System Operator (CAISO) is an enormous, complex energy market system in which energy buyers and sellers schedule a multitude of energy related transactions. The CAISO had some 40 billion dollars in sales, and more than 260,000 gigawatts sold in 2015\cite{CAenergy2}. This makes for a data-rich problem worthy for study by academics and industry professionals alike. A particularly interesting problem in the CAISO is the forecasting of Day-Ahead Market (DAM) prices which are notoriously stochastic. DAM prices are used to forecast energy prices a day in advance which energy buyers and sellers then use to schedule when they will buy and sell energy on the CAISO market. It has been shown that accurate forecasting is absolutely critical to profiting off DAM market prices\cite{Dowling2017}.\\
Recent advances in recurrent neural networks (RNN) for time series learning suggest that RNN's could be a potential way to produce more accurate forecasts of energy prices. In this project, we propose to build a RNN that can accurately predict stochastic energy prices in the CAISO across various energy providers. In order to evaluate this challenge, we will compare against reported literature values of standard time series forecasting such as backcasting and ARIMA predictions\cite{Conejo2005a}. Due to the recent success of RNN's, we expect an RNN to perform as well or better than current methods. 

% ============================================================================================================================= %
%														Approach																       %
% ============================================================================================================================= %

\section{Approach}
Because the dataset contains sequential information in the form of energy prices at different time increments, a recurrent neural network would be well suited to predicting future prices for this problem. This is because unlike in traditional neural networks, in recurrent neural networks there are loops that allow information about previous inputs to persist. In other words, it can use its memory from a previous time step to make a decision in a future time step. This is essential when working with time series data where order and contextual information is extremely important for accuracy.
At a high level, the proposed model would utilize long short-term memory recurrent neural network architecture. Long short-term memory networks are ideal for making time series predictions because they allow the model to decide what information to keep and what information to forget. This allows them to deal with gaps of unknown duration between notable events in a time series.
Our network will accept an input vector of past time series values and then return an output vector of predicted prices over the next 48 hours. 


% ============================================================================================================================= %
%														Data Sources															       %
% ============================================================================================================================= %

\subsection{Data Sources}
The dataset for this project was kindly provided by the Dowling Lab for Uncertainty Quantification and Mathematical Optimization at the University of Notre Dame in the Department of Chemical and Biomolecular Engineering. The dataset is comprised of over 6500 energy vendors, called nodes, participating in the CAISO. There is a full year's worth of data for each node in 2015, and price measurements are recorded at 1 hour intervals. Latitude and longitude coordinates are known for around 2000 of these nodes for visualization purposes.

% ============================================================================================================================= %
%														Evaluation Plan															       %
% ============================================================================================================================= %

\section{Evaluation Plan}

%Instructions:
%\begin{itemize}
%    \item How will you evaluate the proposed work?
%    \item What does success mean for the class project?
%\end{itemize}
%
%Outline:
%\begin{itemize}
%    \item Evaluate vs. literature values.
%    \item Split dataset into training and testing sets.
%    \item Compare CPU time with current methods in literature.
%    \item Success would be building a neural network that forecasts prices of energy markets. 
%    \item Success is not if the NN beats traditional methods, but rather the testing of a new method. 
%\end{itemize}

Fortunately, there is a lot of published work on forecasting DAM markets, so evaluation of our RNN will be somewhat easy to compare with reported literature accuracies. Furthermore, the DAM dataset that was provided is sufficiently large enough to test our RNN's performance on a large amount of geographically separated nodes while still having enough training data to fully learn DAM dynamics. Furthermore, it will be important to evaluate the CPU time needed to train our RNN. CPU time can easily be compared with existing literature values reported. We have determined that success for this project will be defined as the successful implementation of a RNN that forecasts DAM prices in the CAISO. However, as time is somewhat limited for this project, success is not going to be defined as constructing a RNN that performs as well or better than current time-series forecasting methods.  

% ============================================================================================================================= %
%													Implementation Plan															       %
% ============================================================================================================================= %

\section{Project Implementation Plan}
By the first milestone on April 7th, we plan to have our data split into training, testing, and validation sets. We also will solidify the initial architecture of our neural network and begin to implement it on a small scale in a Jupyter Notebook in order to identify roadblocks as soon as possible. 

By the second milestone on April 17th, we plan to extend our implementation across all of the nodes in our data set. We will also perform error analysis to evaluate our work and make any changes necessary from those results. 

By April 27th we will turn in our paper drafts for peer review with detailed analysis of our findings from the first and second milestones.

On May 5th, we will turn in our final paper with edits based on peer feedback and any model developments. This paper will outline the implementation of a neural net for energy forecasting. We will also turn in reproducible code to verify our findings. 

% ============================================================================================================================= %
%														References															       %
% ============================================================================================================================= %

%\section{Related Work}
%\begin{itemize}
%    \item Here provide the relevant references for your work. 
%\end{itemize}
%	
%\emph{Use the standard Communications of the ACM format for references --- that is, a numbered list at the end of the article, ordered alphabetically by first author, and referenced by numbers in brackets~. See the examples of citations at the end of this document. Within this template file, use the style named references for the text of your citation.}
%
%\emph{References should be published materials accessible to the public. Internal technical reports may be cited only if they are easily accessible (i.e. you can give the address to obtain the report within your citation) and may be obtained by any reader. Proprietary information may not be cited. Private communications should be acknowledged, not referenced  (e.g., ``[Robertson, personal communication]").}


\bibliographystyle{ACM-Reference-Format}
%\bibliography{acmart.bib}
\bibliography{/Users/ClayElmore/Desktop/mendeley/DMD_Literature.bib}
\end{document}
