%
% The first command in your LaTeX source must be the \documentclass command.
\documentclass[sigconf]{acmart}

%
% defining the \BibTeX command - from Oren Patashnik's original BibTeX documentation.
\def\BibTeX{{\rm B\kern-.05em{\sc i\kern-.025em b}\kern-.08emT\kern-.1667em\lower.7ex\hbox{E}\kern-.125emX}}
    
% Rights management information. 
% This information is sent to you when you complete the rights form.
% These commands have SAMPLE values in them; it is your responsibility as an author to replace
% the commands and values with those provided to you when you complete the rights form.
%
% These commands are for a PROCEEDINGS abstract or paper.
%\copyrightyear{2018}
%\acmYear{2018}
%\setcopyright{acmlicensed}
%\acmConference[Woodstock '18]{Woodstock '18: ACM Symposium on Neural Gaze Detection}{June 03--05, 2018}{Woodstock, NY}
% \acmBooktitle{Woodstock '18: ACM Symposium on Neural Gaze Detection, June 03--05, 2018, Woodstock, NY}
% \acmPrice{15.00}
% \acmDOI{10.1145/1122445.1122456}
% \acmISBN{978-1-4503-9999-9/18/06}

%
% These commands are for a JOURNAL article.
%\setcopyright{acmcopyright}
%\acmJournal{TOG}
%\acmYear{2018}\acmVolume{37}\acmNumber{4}\acmArticle{111}\acmMonth{8}
%\acmDOI{10.1145/1122445.1122456}

%
% Submission ID. 
% Use this when submitting an article to a sponsored event. You'll receive a unique submission ID from the organizers
% of the event, and this ID should be used as the parameter to this command.
%\acmSubmissionID{123-A56-BU3}

%
% The majority of ACM publications use numbered citations and references. If you are preparing content for an event
% sponsored by ACM SIGGRAPH, you must use the "author year" style of citations and references. Uncommenting
% the next command will enable that style.
%\citestyle{acmauthoryear}

%
% end of the preamble, start of the body of the document source.
\begin{document}

%
% The "title" command has an optional parameter, allowing the author to define a "short title" to be used in page headers.
\title{Energy Market Price Forecasting with Neural Networks}

%
% The "author" command and its associated commands are used to define the authors and their affiliations.
% Of note is the shared affiliation of the first two authors, and the "authornote" and "authornotemark" commands
% used to denote shared contribution to the research.

\author{Elmore, Clay}
%\authornote{Both authors contributed equally to this research.}
\email{celmore1@nd.edu}
\affiliation{%
  \institution{University of Notre Dame}
}

\author{Kopp, Grace}
\email{gkopp@nd.edu}
\affiliation{%
  \institution{University of Notre Dame}
}
%
% By default, the full list of authors will be used in the page headers. Often, this list is too long, and will overlap
% other information printed in the page headers. This command allows the author to define a more concise list
% of authors' names for this purpose.
\renewcommand{\shortauthors}{Elmore and Kopp, et al.}

% ============================================================================================================================= %
%														Abstract																       %
% ============================================================================================================================= %
% The abstract is a short summary of the work to be presented in the article.
\begin{abstract}
This project proposes to construct a neural network that will forecast energy prices in the California Independent System Operator (CISO). Accurate forecasting of energy prices in the CISO provides an economic opportunity for energy providers incur profit and buyers to purchase as the lowest possible price. Traditional methods for forecasting have been used for many years to forecast these prices, but the recent advances in neural network time series forecasting presents a possible way to improve upon existing methods. The problem statement of this project is how to design a neural network to accurately predict 
\end{abstract}

%
% The code below is generated by the tool at http://dl.acm.org/ccs.cfm.
% Please copy and paste the code instead of the example below.
%
% \begin{CCSXML}
% <ccs2012>
%  <concept>
%   <concept_id>10010520.10010553.10010562</concept_id>
%   <concept_desc>Computer systems organization~Embedded systems</concept_desc>
%   <concept_significance>500</concept_significance>
%  </concept>
%  <concept>
%   <concept_id>10010520.10010575.10010755</concept_id>
%   <concept_desc>Computer systems organization~Redundancy</concept_desc>
%   <concept_significance>300</concept_significance>
%  </concept>
%  <concept>
%   <concept_id>10010520.10010553.10010554</concept_id>
%   <concept_desc>Computer systems organization~Robotics</concept_desc>
%   <concept_significance>100</concept_significance>
%  </concept>
%  <concept>
%   <concept_id>10003033.10003083.10003095</concept_id>
%   <concept_desc>Networks~Network reliability</concept_desc>
%   <concept_significance>100</concept_significance>
%  </concept>
% </ccs2012>
% \end{CCSXML}

% \ccsdesc[500]{Computer systems organization~Embedded systems}
% \ccsdesc[300]{Computer systems organization~Redundancy}
% \ccsdesc{Computer systems organization~Robotics}
% \ccsdesc[100]{Networks~Network reliability}

%
% Keywords. The author(s) should pick words that accurately describe the work being
% presented. Separate the keywords with commas.
\keywords{machine learning, forecasting, neural networks, energy markets}

%
% A "teaser" image appears between the author and affiliation information and the body 
% of the document, and typically spans the page. 
% \begin{teaserfigure}
%   \includegraphics[width=\textwidth]{sampleteaser}
%   \caption{Seattle Mariners at Spring Training, 2010.}
%   \Description{Enjoying the baseball game from the third-base seats. Ichiro Suzuki preparing to bat.}
%   \label{fig:teaser}
% \end{teaserfigure}

%
% This command processes the author and affiliation and title information and builds
% the first part of the formatted document.
\maketitle

% ============================================================================================================================= %
%														Intro  																       %
% ============================================================================================================================= %

\section{Introduction}
\begin{itemize}
    \item Provide motivation to the project topic. That is tell the reader about why you want to work on this topic. Set the context and objective of the project here. 
    \item Frame the problem statement and/or the research question for your project.
    \item Summarize the contributions of your work, that is explicitly state what the expected outcomes will be. 
\end{itemize}
	

% ============================================================================================================================= %
%														Approach																       %
% ============================================================================================================================= %

\section{Approach}
\begin{itemize}
    \item Briefly describe your proposed approach here
    \item Discuss what methods you will use and why
\end{itemize}

% ============================================================================================================================= %
%														Data Sources															       %
% ============================================================================================================================= %

\subsection{Data Sources}
\begin{itemize}
    \item Describe your plan to procure the data
    \item Discuss what methods you will use and why
\end{itemize}

% ============================================================================================================================= %
%														Evaluation Plan															       %
% ============================================================================================================================= %

\section{Evaluation Plan}

\begin{itemize}
    \item How will you evaluate the proposed work?
    \item What does success mean for the class project?
\end{itemize}

% ============================================================================================================================= %
%													Implementation Plan															       %
% ============================================================================================================================= %

\section{Project Implementation Plan}
How do you plan to implement your project and achieve the outcomes, including validation plan? Specify your deliverables on different milestones, draft, and deliverable. 


\begin{itemize}
    \item Milestone 1 
    \item Milestone 2 
	\item Paper draft 
	\item Final deliverable 
    \item What does success mean for the class project?
\end{itemize}

% ============================================================================================================================= %
%														References															       %
% ============================================================================================================================= %

\section{Related Work}
\begin{itemize}
    \item Here provide the relevant references for your work. 
\end{itemize}
	
\emph{Use the standard Communications of the ACM format for references --- that is, a numbered list at the end of the article, ordered alphabetically by first author, and referenced by numbers in brackets~\cite{anderson1992social}. See the examples of citations at the end of this document. Within this template file, use the style named references for the text of your citation.}

\emph{References should be published materials accessible to the public. Internal technical reports may be cited only if they are easily accessible (i.e. you can give the address to obtain the report within your citation) and may be obtained by any reader. Proprietary information may not be cited. Private communications should be acknowledged, not referenced  (e.g., ``[Robertson, personal communication]").}


\bibliographystyle{ACM-Reference-Format}
\bibliography{acmart.bib}
\end{document}
